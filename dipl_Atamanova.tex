\documentclass[a4paper,12pt]{amsart}
\usepackage[utf8]{inputenc}
\usepackage[T2A]{fontenc}
\usepackage[english,russian]{babel}
\usepackage{amsfonts, amssymb, amsmath, wrapfig}
\usepackage{longtable}
\usepackage[all]{xy}
\usepackage{epsfig}
\usepackage{verbatim}

\makeatletter
\renewcommand*\@settitle{\begin{center}%
  \baselineskip18\p@\relax
    \bfseries
  \@title
  \end{center}%
}
\renewcommand*\@@and{и}

\def\a{\alpha}
\def\b{\beta}
\def\g{\gamma}
\def\d{\delta}
\def\l{\langle}
\def\r{\rangle}
\def\ph{\varphi}
\def\e{\varepsilon}
\def\o{\omega}
\def\A{\operatorname{A}}
\def\D{\operatorname{D}}
\def\E{\operatorname{E}}
\def\G{\operatorname{G}}
\def\F{\operatorname{F}}
\def\K{\operatorname{K}}
\def\Math{{\tt Ma\-the\-ma\-ti\-ca}}
\def\map{\longrightarrow}
\def\GL{\operatorname{GL}}
\def\SL{\operatorname{SL}}
\def\Aut{\operatorname{Aut}}
\def\Spin{\operatorname{Spin}}
\def\Isom{\operatorname{Isom}}
\def\diag{\operatorname{diag}}
\def\Char{\operatorname{char}}
\def\sic{\operatorname{sc}}
\def\tr{\operatorname{tr}}
\def\GF#1{{\mathbb F}_{\!#1}}
\def\Re{{\mathbb R}}
\def\Co{{\mathbb C}}
\def\Int{{\mathbb Z}}
\def\Nat{{\mathbb N}}
\def\Pr{{\mathbb P}}
\def\height{\operatorname{ht}}
\def\bar{\overline}
\def\tilde{\widetilde}
\def\la{\langle}
\def\ra{\rangle}
\def\fg{\mathfrak g}
\def\dynkin#1#2#3#4#5#6#7#8{\vcenter{\vbox{\vfill\hbox{$#1#3#4#5#6#7#8$}
\nointerlineskip\vskip 3pt
\hbox{$\phantom{#1}\phantom{#3}#2$\hfil}\vfill}}}
\def\smalldynkin#1#2#3#4#5#6#7#8{\vcenter{\vbox{\vfill\hbox{$\scriptstyle #1#3#4#5#6#7#8$}
\nointerlineskip\vskip 3pt
\hbox{$\scriptstyle\phantom{#1}\phantom{#3}#2$\hfil}\vfill}}}

\theoremstyle{plain}
\newtheorem{theorem}{Теорема}
\newtheorem{lemma}{Лемма}

\theoremstyle{remark}
\newtheorem{remark}{Замечание}

\theoremstyle{definition}
\newtheorem{definition}{Определение}

\title[]
{УРАВНЕНИЯ НА ПРИСОЕДИНЕННЫЕ ПРЕДСТАВЛЕНИЯ АЛГЕБРАИЧЕСКИХ ГРУПП И КОМБИНАТОРИКА ПОДСИСТЕМ КОРНЕЙ}
%\author{Атаманова}
\address{Санкт-Петербургский государственный университет}
%\curraddr{Einstein Institute of Mathematics, Hebrew University of Jerusalem}
\email{mariya\_atm@mail.ru}
%\thanks{Автор начинал эту работу в рамках программы DAAD <<Михаил Ломоносов>> и пользовался
%поддержкой грантов РФФИ 09-01-00784, РФФИ 09-01-00878, РФФИ 09-01-90304,
%РФФИ 10-0190016, РФФИ 10-01-00551 и
%Golda Meir Postdoctoral Fellowship.}
\date{}

\begin{document}


\maketitle
\section*{Введение}

Алгебраические группы часто задаются как преобразования, сохраняющие
некоторые полилинейные формы. Так, ортогональная группа
(скажем, над полем характеристики, отличной от $2$) по определению
является группой линейных преобразований, сохраняющих квадратичную форму,
или, что то же самое, сохраняющих соответствующую билинейную форму.
Аналогично, симплектическая группа~--- это группа линейных преобразований,
сохраняющих симплектическую билинейную форму.

В 1905 году Леонард Диксон построил инвариантную кубическую форму для группы
типа $\E_6$. Эта форма от двадцати семи переменных
впоследствии изучалась в работах Клода Шевалле,
Ганса Фрейденталя и многих других. Майкл Ашбахер доказал
(см.~\cite{Aschbacher_e6}), что группа линейных
преобразований 27-мерного пространства, сохраняющих эту форму, совпадает
с односвязной группой Шевалле типа $\E_6$ над произвольным полем (даже в случаях
характеристик 2 и 3)

Еще раньше Леонард Диксон описал инвариантную форму четвертой степени для группы
типа $\E_7$. Эта форма действует на 56-мерном пространстве минимального
представления односвязной группы Шевалле типа $\E_7$. Также известно, что на
этом пространстве есть инвариантная симплектическая форма.
Брюс Куперстейн (\cite{Cooperstein_E7}, см. также~\cite{Aschbacher_multi})
показал, что группа линейных преобразований, сохраняющих обе
этих формы, совпадает с группой Шевалле типа $\E_7$ для случая поля
характеристики, отличной от 2. В работе~\cite{Luzgarev_e7_invariants}
снято ограничение на характеристику
за счет перехода от биквадратной формы к {\em несимметричной} четырехлинейной.

Изучение минимальных представлений групп типов $\E_6$ и $\E_7$ облегчается тем,
что эти представления являются {\em микровесовыми}. У группы типа $\E_8$, в то же
время, вообще нет микровесовых представлений. Ее минимальное представление~---
присоединенное. Поэтому представляет интерес рассмотрение присоединенных
представлений исключительных групп и заданных на них инвариантных полилинейных форм.

Описанные выше полилинейные формы тесно связаны с уравнениями на орбиту вектора
старшего веса. Хорошо известно (см.~\cite{Lichtenstein}), что орбита вектора старшего веса
в любом представлении задается квадратичными уравнениями.
Продифференцируем инвариантную трилинейную форму для группы типа $\E_6$ по каждой
координате: мы получим набор из 27 квадратичных многочленов. Оказывается,
эти многочлены в точности выделяют орбиту вектора старшего веса (над алгебраически
замкнутым полем). Аналогично, вторые частные производные четырехлинейной инвариантной
формы для группы типа $\E_7$~--- квадратичные многочлены, задающие орбиту
вектора старшего веса минимального представления этой группы (впрочем, здесь
возникают тонкости, связанные с наличием симплектической формы).

Уравнения на орбиту вектора старшего веса в присоединенных представлениях
групп типа $\E_6$, $\E_7$, $\E_8$ описаны в статье~\cite{Luzgarev_adjoint}.
Они разбиваются на три
типа, которые в~\cite{Luzgarev_adjoint} названы <<уравнения для угла $\pi/2$>>,
<<уравнения для угла $2\pi/3$>> и <<уравнения для угла $\pi$>>.
Целью настоящей работы является построение кубической формы на пространстве
присоединенного представления группы Шевалле типа $\E_7$, частные производные
которой по любой переменной (кроме переменных, соответствующих нулевым весам),
являются линейными комбинациями уравнений первых двух типов.
Этот результат можно рассматривать как первый шаг на пути к построению
инвариантной полилинейной формы на пространстве присоединенного представления
группы Шевалле типа $\E_7$.

\section{Основные обозначения}
Пусть $\Phi$ --- приведённая неприводимая система корней ранга $l$. Пусть $\Pi={\alpha_1,\ldots,\alpha_l}$ --- фундаментальная система в $\Phi$ (её элементы называются \emph{простыми} корнями). Мы всегда используем ту же нумерацию простых коней, что в \cite{Bourbaki}. Для $\alpha\in\Phi$ мы имеем $\alpha = \sum_{s=1}^{l}m_s(\alpha)\alpha_s$

Пусть $G=G(\Phi,R)$ --- односвязная группа Шевалле типа $\Phi$ над коммутативным кольцом $R$ с $1$. Мы будем работать с присоединенном представлением $G(\Phi,R)$, которое дает нам неприводимое действие $G(\Phi,R)$ на свободном $R$-модуле $V$ ранга $|\Phi|+l$. Обозначим за $\Lambda$ множество весов нашего представления c \emph{кратностями}. То есть $\Lambda = \Lambda^{*}\sqcup\Delta$, где $\Lambda^{*} = \Phi$ --- множество ненулевых весов, а $\Delta = \{0_1, \ldots ,0_l\}$ --- множество нулевых весов. Мы фиксируем допустимый базис $e^{\lambda}, \lambda \in \Lambda$ в $V$. Тогда вектор $v \in V$ может быть единственным образом представлен как $v=\sum_{\lambda \in \Lambda}v_{\lambda}e^{\lambda} = \sum_{\alpha \in \Phi}v_{\alpha}e^{\alpha} + \sum_{i=1}^{l}\hat{v}_{i}\hat{e}^{i}$. Мы часто будем писать просто $v=(v_{\lambda})$.

Множество корней системы $\Phi$ является подмножеством Евклидова пространства $E$ с скалярным произведением $(\cdot,\cdot)$. Мы будем также пользоваться произведением $\langle\alpha,\beta\rangle = 2(\alpha,\beta)/(\beta,\beta)$ для $\alpha, \beta \in E$ (для $\alpha, \beta \in \Phi$ получаем число Картана). Мы будем рассматривать системы корней с \emph{простыми связями}, что означает что все корни длины $1$. Поэтому $\langle\alpha, \beta\rangle = 2(\alpha,\beta)$ для любых $\alpha, \beta \in \Phi$. Обозначим за $\angle(\alpha, \beta)$ угол между $\alpha,\beta \in E$.

Структурные константы $N_{\alpha,\beta}, \alpha,\beta \in \Phi$ простой комплексной алгебры Ли типа $\Phi$ детально описаны в \cite{Vavilov}. Заметим, что в нашем случае $N_{\alpha,\beta} = 0$ или $\pm1$.

В (?) описаны все уравнения на вектор $v=(v_{\lambda})_{\lambda \in \Lambda} \in V$. Выпишем их сюда.
\begin{itemize}
  \item $\pi/2$-уравнения\\
    $S_{\pi/2}(\alpha,\beta)=\{\{\gamma,\delta\} | \gamma + \delta = \alpha + \beta, \{\gamma,\delta\}\neq\{\alpha,\beta\}\}$
    $$v_{\alpha}v_{\beta} = \sum_{\{\gamma,\delta\} \in S_{\pi/2}(\alpha,\beta)}N_{\alpha,-\gamma}N_{\beta,-\delta}v_{\gamma}v_{\delta}$$

  \item $2\pi/3$-уравнения\\
  $S_{2\pi/3}(\alpha,\beta)=\{\{\gamma,\delta\} | \gamma + \delta = \alpha, \{\gamma,\delta\}\neq 0\}$
    $$v_{\alpha}\cdot\sum_{s=1}^{l}\langle\beta,\alpha_s\rangle\hat{v}_s = -\sum_{\substack{\{\gamma,\delta\} \in S_{2\pi/3}(\alpha,\beta)\\ \angle(\gamma,\beta)=\pi/3}}N_{\gamma,\delta}v_{\gamma}v_{\delta}$$

  \item $\pi$-уравнения\\
  $S_{\pi}(\alpha,\beta)=\{\{\gamma,\delta\} | \gamma + \delta = 0, \angle(\gamma,\alpha)=\angle(\gamma,\beta)=2\pi/3\}$\\
  $S_{\pi}'(\alpha,\beta)=\{\{\gamma,\delta\} | \gamma + \delta = 0, \angle(\gamma,\alpha)=\angle(\delta,\beta)=2\pi/3\}$
     $$\sum_{s=1}^{l}\langle\alpha,\alpha_s\rangle\hat{v}_s\cdot\sum_{s=1}^{l}\langle\beta,\alpha_s\rangle\hat{v}_s = \sum_{\{\gamma,\delta\} \in S_{\pi}'(\alpha,\beta)}v_{\gamma}v_{\delta} - \sum_{\{\gamma,\delta\} \in S_{\pi}(\alpha,\beta)}v_{\gamma}v_{\delta}$$
\end{itemize}




\section{Комбинаторика корней}
\subsection{Квадраты}
Напомним основные сведения про квадраты.

Пусть $k = l-1,4,5,7$ для $\Phi = D_l ,E_6 ,E_7 ,E_8$ , соответственно.
\begin{definition}
Множество корней $\Omega=\{\beta_i\},~ i = 1, \ldots ,k,-k, \ldots ,-1$, такие что $\angle(\beta_i,\beta_{-i}) = \pi/2$
для $i = 1, \ldots ,k$ и $\angle(\beta_i,\beta_j) = \pi/3$ для $i \neq \pm j$, называется \emph{максимальным квадратом}.
\end{definition}
Слово максимальный говорит о том, что число $k$ выбрано максимальным. Далее в тексте речь будет идти только о таких квадратах, поэтому слово <<максмальный>>, будет опущено.

Для квадрата сумма корней $\beta_i + \beta_{-i}$ не зависит от $i$, поэтому она общая для всего квадрата. Обозначим этот вектор за $\sigma(\Omega)$. Известно, что пары ортогональных корней с одинаковой суммой задают в точности квадрат.
Более того, по паре ортогональных корней можно восстановить квадрат, взяв все пары корней с той же суммой.

\subsection{Отражение относительно квадрата}
Напомним, что на корнях действует группа Вейля отражением.
$$s_{\alpha}(\beta) = \beta - \langle\alpha,\beta\rangle\alpha$$ 
Так как мы рассматриваем только приведенные кристаллографические системы корней с простыми связями, то $\langle\alpha,\beta\rangle$ может равняться только $2$, $-2$, $1$, $-1$, $0$. А углы между ними будут $0$, $\pi$, $\pi/3$, $2\pi/3$ и $\pi/2$ соответственно. Отсюда видно, что $\angle(\alpha,\beta)=\pi/3$ равносильно тому, что $\alpha-\beta \in \Phi$, а $\angle(\alpha,\beta)=2\pi/3$ равносильно тому, что $\alpha+\beta \in \Phi;$

Рассмотрим квадрат $\Omega$ и отразим множество его корней относительно корня $\alpha$

\begin{definition}
\emph{Отражением квадрата относительно корня $\alpha$} назовем множество $S_{\alpha}(\Omega)=\{s_{\alpha}(\gamma) ~|~ \gamma\in\Omega\}$
\end{definition}

\begin{remark}
\begin{align*}
 &
 1)~ S_{\alpha}(\Omega) \text{ является квадратом.}
\\ &
 2)~ \text{Если } \alpha \in \Omega \text{, то } \sigma(S_{\alpha}(\Omega))=\sigma(\Omega)-2\alpha.
\end{align*}
\end{remark}
\begin{proof}
Пусть $\Omega = \{\beta_1,...,\beta_k, \beta_{-k},...,\beta_{-1}\}$.
Не умаляя общности, $\alpha=\beta_1$. По определению квадрата, $\beta_1$ ортогонален $\beta_{-1}$, а с остальными $\beta_{i}$ образует угол $\pi/3$. То есть отражение подействует следующем образом:
\begin{center}
  $(\beta_1, \ \beta_{-1})\mapsto (-\beta_1, \beta_{-1})$

  $(\beta_i, \beta_{-i})\mapsto (\beta_i-\beta_1, \beta_{-i}-\beta_{1}) \text{ для любого } i\neq\pm1$
\end{center}
Заметим, что сумма у всех получившихся пар совпадает и равна $\sigma(\Omega)-2\alpha$, что дает нам пункт 2). Видно, что первая пара ортогональная. Этот факт с пунктом 2) дает нам пункт 1).
\end{proof}

Заметим, что $\beta_i + \beta_{-i} - 2\beta_1 = \beta_1 + \beta_{-1} - 2\beta_{1} = \beta_{-1} - \beta_{1}$


\subsection{Серии}
Рассмотрим произвольный квадрат $$\Omega = \{\beta_1, \ldots,\beta_k, \beta_{-k}, \ldots,\beta_{-1}\}$$
Будем обозначать $-\Omega$, квадрат, состоящий из обратных корней.
\begin{definition}
Множество квадратов, полученное при отражении квадрата относительно всех его корней, вместе с $\pm\Omega$ называется \emph{серией}, то есть $s(\Omega) = \{S_\alpha(\Omega) | \alpha\in\Omega\}\cup \{\pm\Omega\}$.
\end{definition}

На самом деле, серия, как множество корней, замкнута относительно отражений, а разбиение квадратов на серии задает отношение эквивалентности на них. То есть $s(\Omega)=s$(любого другого представителя этой серии). Таким образом, две серии либо не пересекаются, либо равны. Прежде чем это доказать, изобразим графически серию.

Нарисуем по серии граф, состоящих из помеченных вершин и помеченных ребер. Вершины будут соответствовать квадратам из серии, а ребра будут соответствовать отражениям. То есть один квадрат переходит в другой отражением относительно корня, помеченного на ребре, соединяющем эти квадраты.

По определению серии, вершин будет $2(k+1)$. Пронумеруем их $\mathbf{-k}$, \ldots, $\mathbf{-1}$, $\mathbf{-0}$, $\mathbf{0}$, $\mathbf{1}$, \ldots, $\mathbf{k}$, что будет соответствовать $$S_{\beta_{-k}}(\Omega), \ldots,S_{\beta_{-1}}(\Omega), ~-\Omega, ~\Omega, ~S_{\beta_{1}}(\Omega), \ldots,S_{\beta_{k}}(\Omega)$$
то есть как раз всем квадратам из серии $s(\Omega)$.

Заметим, что $S_{\beta_{-i}}(\Omega) = -S_{\beta_{i}}(\Omega)$, таким образом $\mathbf{i},\mathbf{-i}$ соответствуют противоположным квадратам.

Две ненулевые вершины $\mathbf{i}, \mathbf{j}$ такие, что $|\mathbf{i}| \neq |\mathbf{j}|$, соединим ребром с пометкой $\pm(\beta_{i}-\beta_{j})$.
Ребро $(\mathbf{0} , \mathbf{i})$ пометим $\pm\beta_{i}$, а $(\mathbf{-0}, \mathbf{i})$ пометим $\pm\beta_{-i}$.

\begin{picture}(500,170)
\put(50,150){\textbf{0}} \put(53,153){\circle{17}}
\put(150,150){\textbf{-0}} \put(157,153){\circle{17}}
\put(50,110){\textbf{1}} \put(53,113){\circle{17}}
 \put(150,110){\textbf{-1}}\put(157,113){\circle{17}}
\put(50,70){\textbf{2}} \put(53,73){\circle{17}}
\put(150,70){\textbf{-2}} \put(157,73){\circle{17}}
\put(50,20){\textbf{k}} \put(53,23){\circle{17}}
 \put(150,20){\textbf{-k}} \put(157,23){\circle{17}}
\put(55,145){\line(3,-1){93}} \put(103,130){$\pm\beta_{-1}$}
\put(55,65){\line(3,-1){100}} \put(70,50){$\pm(\beta_{2}-\beta_{-k})$}

\put(53,145){\line(0,-1){25}} \put(53,130){$\pm\beta_1$}
\put(157,145){\line(0,-1){25}} \put(157,130){$\pm\beta_{1}$}
\put(53,105){\line(0,-1){25}} \put(53,90){$\pm(\beta_{1}-\beta_{2})$}
\qbezier(165,150)(200,140)(165,73) \put(180,110){$\pm\beta_2$}
\qbezier(165,150)(250,150)(165,23) \put(180,50){$\pm\beta_k$}

\qbezier(45,150)(-15,140)(45,73) \put(20,110){$\pm\beta_2$}
\qbezier(45,150)(-40,150)(45,23) \put(10,50){$\pm\beta_k$}
\end{picture}
В итоге, получился граф, в котором не соединены только противоположные вершины.
\begin{lemma}
Каждое ребро соединяет квадраты, получающиеся друг из друга отражением относительно пометки на этом ребре.
\end{lemma}
\begin{proof}
Так как отражение идемпотентно, то достаточно доказать утверждение для одного (любого) направления ребра.

Рассмотрим произвольное ребро, если оно выходит из $\mathbf{0}$, то утверждение верно по определению серии.

Посмотрим на ребро из $\mathbf{-0}$. Не умаляя общности, будем рассматривать ребро $(\mathbf{-0},\mathbf{1})$. В квадрате $\mathbf{1}$ есть ортогональная пара $(-\beta_1,\beta_{-1})$, что при отражении относительно $\beta_{-1}$ переходит в $(-\beta_1,-\beta_{-1})$. А эта пара из $\mathbf{-0}$.

Рассмотрим ребро, соединяющее ненулевые вершины, например, $(\mathbf{1},\mathbf{2})$ (для остальных так же, так как все симметрично). В квадрате $\mathbf{1}$ есть ортогональная пара $(\beta_2-\beta_1, \beta_{-2}-\beta_1)$, а в квадрате $\mathbf{2}$ есть ортогональная пара $(\beta_1-\beta_2, \beta_{-1}-\beta_2)$. Заметим, что вторые корни в этих парах совпадают $\beta_{-2}-\beta_1=\beta_{-1}-\beta_2$, так как $\beta_{1}+\beta_{-1}=\beta_{2}+\beta_{-2}$. Таким образом, первая пара переходит во вторую относительно $\beta_1-\beta_2$. И, так как квадрат задается ортогональной парой, то квадрат $\mathbf{1}$ весь переходит в $\mathbf{2}$.
\end{proof}


\begin{remark}
    1) Любые два непротивоположных квадрата из серии пересекаются ровно по одному корню.\\
    2) Две серии либо не пересекаются (по квадратам), либо совпадают.\\
    3) Пометки на ребрах, вышедшие из одной вершины с нужным знаком, являются в точности корнями квадрата этой вершины.\\
    4) Любой корень либо не встречается в серии, либо встречается ровно два раза.\\
\end{remark}
\begin{proof}
1)Рассмотрим ребро, соединяющее два непротивоположных квадрата. Пусть на нем написано $\pm\alpha$. Из доказательства предыдущей леммы видно, что корень, отвечающий пометке на ребре с <<$+$>> входит в квадрат, соответствующий одному концу ребра, а с <<$-$>> в другой. А корень, ортогональный к этим, у них совпадает. Все остальные корни из одного квадрата образуют угол $\pi/3$ c $\alpha$, а из другого $2\pi/3$, и значит, не совпадают. 2)Из симметричности картинки. 3) Из доказательства леммы и пункта 1. 4) Из пункта 1 и пункта 3.
\end{proof}


\begin{lemma}
Серия замкнута относительно отражения квадратов корнями из серии.
\end{lemma}
\begin{proof}
В силу рассмотренного графа, достаточно рассмотреть только отражения квадрата относительно корней, не входящих в этот квадрат. Рассмотрим квадрат $\mathbf{1}$ и рассмотрим его отражения относительно $\beta_2$. В $\mathbf{1}$ есть ортогональная пара $(-\beta_1,\beta_{-1})$ отразив ее относительно $\beta_2$ получаем $-\beta_1+\beta_2,\beta_{-1}-\beta_2$. Заметим, что сумма сохранилась.  То есть весь квадрат перейдет при таком отражении в себя. Мы рассмотрели конкретный пример, но в силу симметрии картинки этого достаточно.

\end{proof}


Поскольку квадрат задается двумя ортогональными корнями $\alpha, \beta$, то в некоторых удобных для нас случаях серию будем обозначать как  $s(\alpha,\beta)$. Также, иногда будем воспринимать это множество как множество квадратов, а иногда как множество корней, участвующих в этих квадратах. Однако, из контекста всегда будет понятно, что имеется в виду.
\\


Посмотрим на серию как на множество корней. Заметим, что по построению серия замкнута относительно отражения и конечна, значит - это система корней. Несложным вычислением можно проверить, что в одной серии $2k(k+1)$ корней. По классификации мы можем однозначно понять, что это за система. Оформим это в виде леммы

\begin{lemma}
Множество корней серии из  $E_6$, $E_7$, $E_8$ является $D_5$, $D_6$, $D_8$ соответственно.
\end{lemma}

Посмотрев на расширенные диаграммы Дынкина, заметим, что это максимальные вложения типа  $D_l$.

Напомним, что у системы $D_l$ у ортогональной пары корней две орбиты. За счет этого, квадраты разделяются на длинные и короткие, а именно, длиной $2l-2$ и $6$. Квадраты серии отвечают <<длинным>> квадратам $D_l$.


Напомним, что корень называется \emph{максимальным}, если у него максимальная сумма коэффициентов в разложении на простые корни. Он существует и единственен. Далее будем обозначать за $\rho$ максимальный корень в $E_8$.  Систему $E_7$ будем рассматривать как подмножество корней в $E_8$ с нулевым коэффициентом при простом корне $\alpha_8$.


Будем называть $\emph{типом}$ веса его коэффициент перед $\alpha_8$. Таким образом, максимальный корень, например, типа $2$.
Заметим, что тип $\lambda$ может быть только $0, \pm1, \pm2$, в случаях $\lambda \in E_7$, $\lambda\neq\pm\rho \notin E_7$, $\lambda=\pm\rho$, соответственно. Будем называть $\emph{типом}$ квадрата $\Omega$ коэффициент перед $\alpha_8$ у задающей суммы $\sigma(\Omega)$

\begin{remark}
Тип квадрата может быть только $0,\pm1,\pm2$.
\end{remark}
\begin{proof}
Так как $\sigma$ это сумма двух корней, то по предыдущему предложению, коэффициент может принимать от $-4$, до $4$. Значения $\pm3,\pm4$, можно добиться только используя $\pm\rho$. Но тогда в квадрате будет не больше двух корней ?!?
\end{proof}

Заметим, что типы квадратов из одной серии имеют одинаковую четность (так как при отражении сумма отличается на $2c$). Таким образом, все серии можно разделить на $\emph{четные}$ и $\emph{нечетные}$.

Будем говорить, что серия $s$ из $E_8$ $\emph{содержит}$ серию из $E_7$, если при ограничении ее на корни из $E_7$, получится серия из $E_7$. Легко заметить, что такая серия четная, так как в нее входят квадраты типа $0$. Серию, полученную при ограничении, будем обозначать $s|_{E_7}$.

И, аналогично про квадрат будем говорить, что он из $E_7$, если при ограничении получится квадрат из $E_7$.
Заметим, что по построению серия, содержащая квадрат из $E_7$, содержит серию из $E_7$.

\begin{lemma}
Пусть $\rho$ --- максимальный элемент $E_8$, $s$ --- произвольная серия из $E_8$. Серия $s$ содержит серию из $E_7$ тогда и только тогда, когда $\rho \in s$.
\end{lemma}
\begin{proof}
Пусть $\rho \in s$, тогда серия четная. Рассмотрим квадрат, который содержит $\rho$, он типа $2+0$. Значит, остальные ортогональные пары этого квадрата типа $1+1$. Отразим наш квадрат относительно корня типа $1$. Тогда первая пара перейдет в $1+(-1)$, а остальные в $0+0$. Полученные пары корней типа  $0+0$ составляют квадрат из $E_7$. А значит, $s$ содержит серию из $E_7$.


Пусть $s$ содержит серию из $E_7$. Рассмотрим квадрат из $E_7$, содержащийся в серии $s$, он типа $0$. Пар типа $0+0$ в нем $2(k-1)$, рассмотрим оставшиеся две пары. Так как сумма типов корней в этих парах равна $0$, то они могут быть только либо $2+(-2)$, либо $1+(-1)$. Но $2+(-2)$ это $\pm\rho$, то есть не ортогональные. Значит, оставшиеся две пары типа $1+(-1)$. Тогда, отразив эти две пары относительно корня типа $-1$, получим следующие суммы $2+0$, $1+1$. А корень типа $2$, только $\rho$.
\end{proof}

Заметим, что парный корень с $\rho$ в такой серии из $E_7$. Таким образом, любая серия параметризуется корнем $\alpha$ из $E_7$. А именно, $\alpha \leftrightarrow s(\rho,\alpha)$.

\begin{lemma}
Пусть $\alpha$ --- произвольный корень из $E_7$, тогда
    $$\{\gamma\in E_7|~ \gamma \bot \alpha \} = \{\gamma\in E_7|~ \gamma \in s(\rho,\alpha)|_{E_7}\}$$
\end{lemma}
\begin{proof}
Пусть $\beta \in s(\rho,\alpha)|_{E_7}$ не ортогонален $\alpha$.
Заметим, что $\rho \bot \beta$, так как $\rho \bot E_7$. Отразим $\Omega(\rho,\alpha)$ относительно $\beta$, тогда получится квадрат, содержащий $\rho$, и неравный при этом $\Omega(\rho,\pm\alpha)$, но в серии один корень встречается ровно два раза. ?!?.
Количество корней в нашей серии равно количество корней в $D_6$, то есть 60. Количество ортогональных корней к данному в $E_7$ ровно столько же, значит множества совпадают.
\end{proof}

\begin{comment}
\begin{lemma}
Пусть $\alpha \bot \beta \in E7$, тогда $\alpha \in s(\rho, \beta) \Leftrightarrow \beta \in s(\rho, \alpha)$
\end{lemma}
\begin{proof}
Достаточно доказать только в одну сторону (потому что все симметрично).
\end{proof}

Теперь мы все подготовили, чтобы доказать главную лемму (которая доказывает соответствие корней в форме).
\begin{lemma}
Пусть $\beta \bot \gamma \in s(\rho,\alpha)|_{E7}$. Тогда $\gamma, \alpha \in s(\rho,\beta)|_{E7}$ и $\alpha, \beta \in s(\rho,\gamma)|_{E7}$
\end{lemma}
\begin{proof}
\end{proof}
\end{comment}

\section{трилинейные формы на $E_7$}

\subsection{$\pi/2$-форма}

Пусть $\rho$ - старший корень $E_8$. Пусть $c_{(\alpha,\beta,\gamma)}=\pm1$.
$$F=\sum_{\substack{\alpha\bot\beta\bot\gamma\bot\alpha\in E_7 \\ {\rho,\alpha,\beta,\gamma}\in D_4}}c_{(\alpha,\beta,\gamma)}v_{\alpha}v_{\beta}v_{\gamma}$$

Рассмотрим $F$ как многочлен от $133$ переменных над $Z$.
Обозначим $\frac{\partial F}{\partial v_{\alpha}}$ как производную этого многочлена по $v_{\alpha}$.
\begin{theorem}
Существует выбор знаков $c_{(\alpha,\beta,\gamma)}$ такой, что:
    \begin{multline*}
        \frac{\partial F}{\partial v_{\alpha}} = \text{линейная комбинация уравнений},\\
         \text{построенных по квадратам из серии } s(\rho, \alpha)|_{E_7}
    \end{multline*}
\end{theorem}
\begin{proof}
Доказательство состоит из двух частей: соответствие корней, соответствие знаков. 

1) Соответствие корней:
 
 Рассмотрим дифференцирование по одной переменной, например, по $v_\alpha$. Докажем, что слагаемые соответствуют $s(\rho,\alpha)|_{E_7}$. По Лемме 5 $\alpha ~\bot~ s(\rho,\alpha)|_{E_7}$. Чтобы доказать, что четверка попарно ортогональных корней принадлежит $D_4$, нужно найти корень такой, что вся четверка имеет угол с ним $\pi/3$  или $2\pi/3$. Рассмотрим ортогональную пару $(\beta,\gamma)$ из квадрата, принадлежащего серии $s(\rho,\alpha)$. Так как квадраты построенные по $(\beta,\gamma) \text{ и } (\rho,\alpha)$ из одной серии, у них есть общий корень (по замечанию 2), назовем этот корень $\omega$. Заметим, что $\angle(\rho/\alpha/\beta/\gamma,\omega)=\pi/3$, так как $\omega$ со всеми ними лежит в квадратах и не парный им. Значит $(\rho,\alpha,\beta,\gamma)\in D_4$. Тем самым мы доказали, что все слагаемые из серии содержатся. Осталось доказать, что нету лишних слагаемых.

Пусть $\alpha,\beta,\gamma$ тройка, соответствующая слагаемому из F. Докажем, что квадрат $(\beta,\gamma) \in s(\rho,\alpha)$. Так как корни из $D_4$, существует $\omega$, образующая с каждым из них угол $\pi/3$, значит $\omega$ принадлежит квадрату $(\beta,\gamma)$ и квадрату $(\rho,\alpha)$. То есть теперь, чтобы доказать, что эти квадраты из одной серии, нужно доказать, что противоположные корни к $\omega$ --- обратные. Вычислим их из равенств: $\rho + \alpha = \omega + (\rho + \alpha - \omega), ~ \beta + \gamma = \omega + (\beta + \gamma - \omega)$. То есть нужно проверить, что $\rho + \alpha - \omega = -(\beta + \gamma - \omega)$, то есть что $\rho+\alpha+\beta+\gamma = 2\omega$, а это верно для $D_4$.

2) Соответствие знаков:

 Заметим, что, зафиксировав знак у одного слагаемое, мы фиксируем знак у всех слагаемых трех уравнений, пересекающихся по этому слагаемому. Полученные слагаемые вновь с кем-то пересекаются и так далее. Дойдем ли мы так до любого слагаемого из $F$? На самом деле, нет, то есть F распадется на несколько кусочков отдельно для которых тоже выполнено условие согласованности корней теоремы. Запишем это более формально.
 
 Рассмотрим слагаемое $v_{\alpha}v_{\beta}v{\gamma}$. Рассмотрим сумму корней $\delta=\alpha+\beta+\gamma$.  Заметим, что, переходя по слагаемым от одного уравнения к другому, $\delta$ не меняется (по определению квадрата). Таким образом, каждая <<часть>> $F$ параметризуется вектором = сумме корней каждого слагаемого. Из определения $F$ видно, что для каждого слагаемого $\rho+\alpha+\beta+\gamma = 2\omega$, где $\omega$ типа 1. Таких корней $56$, а значит, $F$ распалась на $56$ кусочков. Можно заметить, что в каждом куске по $45$ слагаемых, то есть на самом деле, $F$ равна следующему:

 $$F = \sum_{V(\varpi_6)\subset E_7}(\text{трилинейная форма для } E_6)$$
А так как известно, что на $V(\varpi_6)$ (микровесовом представлении $E_6$) такая форма существует, причем единственная, то и у $F$ можно подобрать знаки так, чтобы теорема была верна, причем $2^{56}$ способами.
\end{proof}


\subsection{$2\pi/3$-форма}

\begin{definition}
    Назовем тройку корней $\alpha,\beta,\gamma$ ---  \emph{$2\pi/3$-тройкой}, если её корни можно переименовать так, что $\angle(\alpha,\beta)=2\pi/3$,  $\angle(\alpha,\gamma)=\pi/2$, $\angle(\gamma,\beta)=\pi/2$.
\end{definition}
Множество $2\pi/3$-троек в $E_7$ назовем $\Upsilon$
Пусть $c_{\alpha,\beta,\gamma}=\pm1$

$$G = \sum_{\substack{(\alpha, \beta, \gamma) \in \Upsilon}}c_{\alpha,\beta,\gamma}v_{\alpha}v_{\beta}v_{\gamma} + \sum_{\substack{\alpha \bot \beta \bot \gamma\bot\alpha\in E7\\{\rho,\alpha,\beta,\gamma}\in D_4}}
c_{\alpha,\gamma,\hat{v}}v_{\alpha}v_{\gamma}\sum_{s=1}^{l}\langle\beta,\alpha_s\rangle\hat{v}_s $$

$$F_1 = \sum(\text{сумма }2\pi/3\text{-уравнений из }D_6)(\text{корень}\bot D_6)$$

\begin{theorem}
Существует выбор знаков  $c_{\alpha,\beta,\gamma}$, такой что:\\
а) $G = F_1$.\\
б) Пусть $\alpha$ --- ненулевой корень.
    \begin{multline*}
        \frac{\partial G}{\partial v_{\alpha}} = \text{линейная комбинация $\pi/2$ и $2\pi/3$-уравнений},\\
         \text{построенных по квадратам из серии } s(\rho, \alpha)|_{E_7}
    \end{multline*}
\end{theorem}
\begin{proof}
Доказательство  опять  же состоит  из  двух  частей:  соответствие  корней,  подбор  знаков.

1) Заметим, что все слагаемые $G$, не содержащие нулевые веса, соответствуют $2\pi/3$-тройке. И все $2\pi/3$-тройки соответствуют какому-нибудь слагаемому $G$ по лемме 5.

Докажем соответствие корней.
Рассмотрим корни $\alpha,\beta,\gamma$, отвечающие какому-нибудь слагаемому. Зафиксируем $\beta$. Продифференцировав $F_1$ по $v_\beta$, мы получим уравнение в которое будет входить слагаемое $v_{\alpha}v_{\gamma}$, а значит, должны входить и слагаемые, отвечающие остальным ортогональным парам квадрата $(\alpha,\gamma)$. Рассмотри этот квадрат $\Omega = \{\beta_1, \ldots, \beta_{-1}\}$. Чтобы проверить соответствие корней, нужно проверить, что $(\beta_i,\beta_{-i},\beta)$ $2\pi/3$-тройка.

Заметим, что сумма у этих троек равны, назовем ее $\omega=\alpha+\beta+\gamma$. Заметим, что $\langle\omega,\beta\rangle= 0-1+2 = 1$. Таким образом, $\langle\beta_i,\omega\rangle + \langle\beta_{-i},\omega\rangle = -1$. Это может быть только $-1+0$ или $-2+1$. Но $-2+1$ означает, что $\beta_i=-\beta$ и $\angle(\beta_{-i},\beta)=\pi/3$, а такого быть не может так как $\beta_i \bot \beta_{-i}$. Таким образом, реализуется только $-1+0$, что значит, что $\angle(\beta_i,\beta)=2\pi/3$, $\angle(\beta_{-i},\beta)=\pi/2$. Таким образом, $(\beta_{i},\beta_{-i},\beta)$  $2\pi/3$-тройка.

2) Знаки подобраны  при  помощи  вычисления  на  компьютере. Поясним  как это сделано. Как и в прошлой теореме заметим, что $G$ распадется на куски, для каждого из которых выполнено условие соответствия корней теоремы.
 $$G=\sum{G_i}~ \text{, где }~ G_i=\sum_{\alpha+\beta+\gamma = fix}v_{\alpha}v_{\beta}v_{\gamma}$$
 Чтобы посчитать количество $G_i$, нужно понять, чему может быть равна сумма $2\pi/3$-тройки. Пусть $\alpha,\beta,\gamma$ --- $2\pi/3$-тройка. Заметим, что $\delta = \alpha+\beta \in \Phi$, так как образуют угол $2\pi/3$. Заметим, что $\langle\delta,\gamma\rangle = 0$ по линейности. То есть $\angle(\delta,\gamma)=0$. А, так как они могут быть любыми ортогональными, то количество разных их сумм, это в точности количество квадратов в $E_7$, то есть $756$. Таким образом, $G$ распадается на $756$ кусочков, каждое из которых выглядит как сумма слагаемых, отвечающих $2\pi/3$-тройкам с одинаковой суммой корней. 
 
 Ясно, что чтобы проверить, то, что можно выбрать знаки для всей $G$, достаточно проверить, что можно выбрать знаки для какой-нибудь (а значит и для всех) $G_i$. А в любой $G_i$ всего лишь $80$ слагаемых. И знаки, при фиксированном одном слагаемом однозначно восстанавливаются. Таким образом, выбрать знаки у $G$ можно, и, даже не одним способом, а $2^{756}$.
\end{proof}



\begin{comment}
\begin{definition}
    Назовем тройку корней $\alpha,\beta,\gamma$ $\pi$\emph{-тройкой}, если корни можно переименовать так, что $\angle(\alpha,\beta)=\pi$,  $\angle(\alpha,\gamma)=\pi/2$, $\angle(\gamma,\beta)=\pi/2$.
\end{definition}
Множество $\pi$-троек в $E_7$ назовем $\Pi$


$$F_2 = \sum(\text{сумма }\pi\text{-уравнений из }D_6)(\text{корень}\bot D_6) + \sum_{(\alpha,\beta,\gamma) \in A_3}v_{\alpha}v_{\beta}v_{\gamma}$$

$$F_2 = \sum_{(\alpha, \beta, \gamma) \in \Pi}v_{\alpha}v_{\beta}v_{\gamma} + \sum_{(\alpha,\beta,\gamma) \in A_3}v_{\alpha}v_{\beta}v_{\gamma} + \sum{v_{}}$$
\begin{theorem}
Пусть $\alpha$ --- ненулевой корень.
    \begin{multline*}
        \frac{\partial F_2}{\partial v_{\alpha}} = \text{линейная комбинация $\pi/2$, $\pi$ и $2\pi/3$-уравнений},\\
         \text{построенных по квадратам из серии } s(\rho, \alpha)|_{E_7}
    \end{multline*}
\end{theorem}
\begin{proof}
Рассмотрим слагаемое, соответствующее $\pi$-тройке $\alpha,\beta,\gamma$.
\end{proof}

\end{comment}

\begin{thebibliography}{99}
\def\selectlanguageifdefined#1{
\expandafter\ifx\csname date#1\endcsname\relax
\else\language\csname l@#1\endcsname\fi}


\bibitem{Luzgarev_e7_invariants}
\selectlanguageifdefined{russian}
А.~Ю. Лузгарев, \emph{Не зависящие от характеристики инварианты четвертой степени
для {$G(\E_7,R)$}}, Вестник {С}анкт-{П}етербургского государственного
университета. {С}ерия 1: {М}атематика. {М}еханика. {А}строномия (2013),
{\cyr\textnumero}~1, 44--51.

\bibitem{Bourbaki}
\selectlanguageifdefined{russian}
Н. Бурбаки, \emph{Группы и алгебры Ли.} Гл. {\rm IV -- VI}, Мир, М., (1972)

\bibitem{Vavilov}
\selectlanguageifdefined{russian}
Н.~А. Вавилов \emph{Как увидеть знаки структурных констант?}

\bibitem{Aschbacher_multi}
\selectlanguageifdefined{english}
M.~Aschbacher, \emph{Some multilinear forms with large isometry groups}, Geom.
  Dedicata \textbf{25} (1988), no. 1-3, 417--465.

\bibitem{Cooperstein_E7}
\selectlanguageifdefined{english}
B.~N. Cooperstein, \emph{The fifty-six-dimensional module for {$\E_7$}. {I}.
  {A} four form for {$\E_7$}}, J. Algebra \textbf{173} (1995), no.~2, 361--389.

\bibitem{Lichtenstein}
\selectlanguageifdefined{english}
W.~Lichtensein, \emph{A system of quadrics describing the orbit of the highest
  weight vector}, Proc. Amer. Math. Soc \textbf{84} (1982), no.~4, 605--608.

\bibitem{Luzgarev_adjoint}
\selectlanguageifdefined{english}
A. Luzgarev, \emph{Equations determining the orbit of the highest weight vector
in the adjoint representation}, {\tt arXiv:1401.0849 [math.AG]}.

\bibitem{Aschbacher_e6}
M. Aschbacher, \emph{The $27$-dimensional module for $\E_6$. {\rm I - IV}},
Invent. Math. \textbf{89} (1987), no.~1, 159--195;
J. London Math. Soc. \textbf{37} (1988), 275--293;
Trans. Amer. Math. Soc. \textbf{321} (1990) 45--84;
J. Algebra \textbf{191} (1991) 23--39.
\selectlanguageifdefined{english}


\end{thebibliography}


\end{document}

